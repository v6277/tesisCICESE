\chapter{Resultados}\label{capit:cap3}
\vspace{-2.0325ex}%
\noindent
\rule{\textwidth}{0.5pt}
\vspace{-5.5ex}% 
\newcommand{\pushline}{\Indp}% Indent puede ir o no :p

Esta es la sección más importante de la tesis. Se sugiere presentar los resultados en el mismo orden que la metodología. Se describen con la ayuda de tablas, gráficas o figuras, de manera que faciliten su comprensión. Las tablas se utilizan generalmente para comunicar valores concretos de los datos, mientras que las figuras son mejores para mostrar tendencias o relaciones entre variables. Se recomienda utilizar subtítulos para agrupar resultados similares y para separar resultados de diferentes parámetros. Una regla básica es no duplicar la información de una tabla en gráficas. Si los mismos resultados de una tabla o figura dentro del contenido de la tesis proporcionan información relevante, se recomienda ponerla como anexo. 

\begin{table}[ht]
\centering
\resizebox{\textwidth}{!}{
\begin{tabular}{||c | c | c ||}
\hline
\hline
Categor\'ia  & Comportamiento  & Definici\'on  \\
\hline
\multirow{5}{4cm}{Actividad fisca. \\ Los movimientos que el niño realiza durante el juego} & Estacionario  & Mantenerse inmóvil durante 3 segundos o
más. Movimientos de dedos o pies\\
\cline{2-3}
& Movimiento extremidades & Movimientos del tronco, brazos y piernas sin mover todo el cuerpo de un lugar a otro\\
\cline{2-3}
& Translocacion lento & Mover el cuerpo de un lugar a otro con una velocidad lenta.\\
\cline{2-3}
& Translocacion medio & Mover el cuerpo de un lugar a otro con una velocidad moderada.\\
\cline{2-3}
& Translocacion ra\'pido & Mover el cuerpo de un lugar a otro con una velocidad rápida\\
\hline
\end{tabular}
}
\caption{Esquema de codificación.}
\label{codschema}
\end{table}

	
\newpage
%%=====================================================

