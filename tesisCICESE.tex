% La compilacion se debe hacer con la instruccion de PdfLatex
% para evitar problemas de tamaño de hoja.
% Todos los archivos de figuras deben estar en formato PDF.
% Al momento de imprimir el documento, se debe hacer sin escala de pagina.
\documentclass[letterpaper,11pt]{cicese}
%%%%%%%%%%%%%%%%%%%%%%%%%%%%%%%%%%%%%%%%%%%%%%%%%%%%%%%%%%%%%%%%%%%%%%%%%%%%%%%%%%%%%%%%%%%%%%%%%%%%%%%%%%%%%%%%%%%%%%%%%%%%%%%%%%%%%%%%%%%%%%%%%%%%%%%%%%%%%%%%%%%%%%%%%%%%%%%%%%%%%%%%%%%%%%%%%%%%%%%%%%%%%%%%%%%%%%%%%%%%%%%%%%%%%%%%%%%%%%%%%%%%%%%%%%%%
\usepackage{marvosym}
\usepackage[letterpaper,left=3cm,right=2cm,top=2cm,bottom=2cm]{geometry}
%uso de acentos directamente en el codigo de LaTeX, sin barra
%\usepackage[ansinew]{inputenc}  
%\usepackage[T1]{fontenc}
%\usepackage[spanish,english]{babel}
%\usepackage[spanish,mexico,es-lcroman]{babel}
%paquete para insertar hipervinculos en referencias a capitulos, figuras, etc.
%\usepackage{arev}
%uso de fuente por defecto de LaTeX para las ecuaciones
%\usepackage{newtxmath} 
\usepackage{amsmath}
\renewcommand{\familydefault}{\sfdefault} 
\usepackage[spanish, mexico,es-lcroman]{babel}
\usepackage[utf8x]{inputenc}
\usepackage{mdframed}
\usepackage[acronym]{glossaries}
\makeglossaries
\usepackage[linkcolor=blue]{hyperref}
\usepackage{cicese}
\usepackage{tabularx}
\usepackage{pbox}
\usepackage{makeidx}
\usepackage{eurosym}
\usepackage{amssymb}
\usepackage{amsmath}
\usepackage{gensymb}
\usepackage{graphicx} % figuras
\graphicspath{{figuras//}}
\usepackage{emptypage} 
\usepackage{textcomp}
\usepackage{physics}

%\usepackage[linesnumbered,ruled,vlined]{algorithm2e}
\DeclareMathAlphabet{\mathantt}{OT1}{antt}{li}{it}
\DeclareMathAlphabet{\mathpzc}{OT1}{pzc}{m}{it}
\setlength{\parskip}{12pt plus 1pt minus 1pt}


% Paquete para pseudocodigo o algoritmos
%\usepackage[lined,boxed,boxruled,linesnumbered,spanish]{algorithm2e}
% Paquete para las graficas
\usepackage{float}
% Paquete para las enumeraciones
\usepackage{enumerate}
% Paquete para pseudocodigo o algoritmos
%este es el que venía:
%\usepackage[lined,boxed,boxruled,linesnumbered,spanish]{algorithm2e}

\usepackage{titlesec}
%\titleformat{\chapter}[display]
%  {\normalfont\huge\bfseries}{\chaptertitlename\ \thechapter}{16pt}{\Huge}

  \titleformat{\chapter} % command
	[hang] % shape
	{\normalfont\bfseries\Large} % format
	%{\chaptertitlename\ \thechapter .} % label
	{\ifnum\pdfstrcmp{\chaptertitlename}{Cap\'itulo}=0  
			\chaptertitlename\ \thechapter .
		\fi
	} % label
	{
	16pt
	} % sep
	{
	%    \rule{\textwidth}{1pt}	
	%    \vspace{1ex}
	%    \centering
	} % before-code
	[
		\vspace{-2ex}%
	  %\rule{\textwidth}{0.3pt}
  ] % after-code


%\titlespacing{command}{left spacing}{before spacing}{after spacing}[right]
\titlespacing\chapter{0pt}{-40.25pt plus 1pt minus 2pt}{12pt plus 2pt minus 2pt}
\titlespacing\section{0pt}{6pt plus 5pt minus 2pt}{6pt plus 2pt minus 2pt}
\titlespacing\subsection{1pt}{6pt plus 4pt minus 2pt}{4pt plus 2pt minus 2pt}
\titlespacing\subsubsection{0pt}{6pt plus 4pt minus 2pt}{4pt plus 2pt minus 2pt}

  	
% Paquete para estilizar los captions de tablas e imagenes
\usepackage{float}
\floatstyle{plaintop}
\restylefloat{table}
\usepackage[font=footnotesize,format=plain,labelfont=bf,up,textfont=normalfont,md,up,justification=justified,labelsep=period]{caption}
\captionsetup{labelfont=bf}

% Para cambiar el color de fondo de una celda en una tabla
\usepackage[table]{xcolor}	
% Para rotar una tabla
\usepackage{rotating}
%\usepackage{subfig}
\usepackage{subfigure} % subfiguras
% Asignar espacio personalizado
\usepackage{setspace}
% Para hacer tablas de multiples páginas
\usepackage{longtable}
% Para utilizar multiples renglones en tablas
\usepackage{multirow}
\usepackage{booktabs}


%%%%%%%%%%%%%%%%%%%%%%%%%%%%%%%%%%%%
%para agregar teoremas, etc
\newtheorem{theorem}{Teorema}[chapter]
\newenvironment{proof}{\noindent{\bf Demostraci\'on.}}{\hfill$\blacksquare$} 
\newtheorem{lemma}{Lema}[chapter]
\newtheorem{teo}{Teorema}[chapter]
\newtheorem{obs}{Observaci\'on}[chapter]
\setcounter{MaxMatrixCols}{10}

\input{tcilatex}
\setcounter{secnumdepth}{3}
\setcounter{tocdepth}{3}
\begin{document}

% Escribe tu nombre, tal y como aparece en los registros
% de Posgrado. De ser necesario, separa con \- las sílabas
% de tu nombre
\autor{Nombre completo del estudiante}

% Escribe el título de tu tesis 
\titulo{Título de la tesis}
%Si el titulo es mayor de 2 renglones es necesario cambiar el valor de espacio vertical 
% para que conserve las dimensiones del margen.
% (\dos) indica que el titulo es de dos renglos en caso de que sean tres 
% usar (\tres)
\dos

% Para el resumen, escribe aqui el título en inglés
\tituloIN{Title of thesis}

% Aquí va el nombre de tu programa de posgrado
\posgrado{nombre del posgrado}
\posgradoUP{NOMBRE DEL POSGRADO MAYUSCULAS}
% Aquí va el nombre de orientacion para el que aplique (sino comentarlo)
\orientacion{ con orientaci\'on en ...}

% Escribe aquí el nombre del posgrado en inglés
\posgradoIN{Name of the Degree}
% Escribe aquí el nombre de orientacion para el que aplique (sino comentarlo)
\orientacionIN{ with orientation in ...}

% Este comando especifica si la tesis es de maestria (usa \mc) o de
% doctorado (usa \dc)
\mc

% Fecha (mes y año) de la defensa de tesis, en español y en ingles
\fechaexamen{mes, a\~no}\fechaexamenIN{month, year}

% Fecha (dia, mes y año) de la defensa de tesis para la hoja de firmas
\fechaexamencompleta{mes, a\~no}

% Año de la defensa //Este valor a aparecerá en la parte inferior de la primer página de la tesis
\anioDefensa{año de defensa}

% Nombre de tu director de tesis, si tienes codirector repite el comando
\director{Dr./Dra. Xxxx Xxxx Xxxx}
%\director{Dr./Dra. Xxxx Xxxx Xxxx}

% Nombres de los sinodales, en el orden que aparecerán en la
% página de firmas
%%NO UTILIZAR ABREVIATURAS
\sinodal{Dr./Dra. Xxxx Xxxx Xxxx}
\sinodal{Dr./Dra. Xxxx Xxxx Xxxx}
\sinodal{Dr./Dra. Xxxx Xxxx Xxxx}
%\sinodal{Dr. 4}
%\sinodal{Dr. 5}
%\sinodal{Dr. 6}

% Para mostrar el logotipo de su programa, elimine el simbolo de comentario del nombre del posgrado correspondiente
%\logotipo{0} %% Ciencias de la vida
%\logotipo{1} %% Acuicultura
%\logotipo{2} %% Ecologia marina
%\logotipo{3} %% Oceanografia marina
%\logotipo{4} %% Ciencias de la computacion
%\logotipo{5} %% Electronica y telecomunicaciones
%\logotipo{6} %% Optica
%\logotipo{7} %% Ciencias de la tierra
\logotipo{8} %% Nanociencias
%\logotipo{9} %% Tecnologias avanzadas e integradas
% Nombre del Coordinador del Programa de Posgrado
\cpp{Dr./Dra. Xxxx Xxxx Xxxx}

% Nombre del Director de Estudios de Posgrado
\dirposgrado{Dr. Pedro Negrete Regagnon}
% Escribe aqui el nombre del archivo .tex que contiene el texto del
% resumen en español. Omite escribir la extensión .tex
\resumenES{secciones/Resumen}

% Palabras clave del resumen en español
\palabrasclave{pc1, pc2, ..., pc5}

% Escribe aqui el nombre del archivo .tex que contiene el texto del
% resumen en inglés. Omite escribir la extensión .tex
\resumenIN{secciones/Abstract}

% Palabras clave del resumen en inglés
\keywords{k1, k2, ..., k5}

% Nombre del archivo .tex con el texto de la dedicatoria, omite
% la extensión .tex
\dedicatoria{secciones/Dedicatoria}

% Nombre del archivo .tex con el texto de los agradecimientos, omite
% la extensión .tex
\agradecimientos{secciones/Agradecimientos}

% Este comando crea todas las páginas preliminares de la tesis
\preliminares

{\normalsize
\chapter{Introducci\'on esta es una prueba para que el título se pueda hacer en dos renglones}\label{capit:cap1}
\vspace{-2.0325ex}%
\noindent
\rule{\textwidth}{0.5pt}
\vspace{-5.5ex}% 
\newcommand{\pushline}{\Indp}% Indent puede ir o no :p

La introducción es la presentación del trabajo, informa sobre tres elementos muy importantes de la investigación: el propósito, la importancia del trabajo realizado y el conocimiento actual del tema. El texto debe comenzar con consideraciones generales y se recomienda terminar con el propósito del trabajo. Es conveniente que sea breve (no mayor de cinco páginas) y debe existir coherencia entre las distintas secciones que se presentan. La redacción debe ser clara, directa y sencilla, de tal forma que un lector no familiarizado con el tema pueda comprender el alcance del trabajo y motive su lectura.

\section{Antecedentes}\label{secc:antece}
En esta sección se profundiza en el conocimiento acerca del tema y la relación con el trabajo de investigación. Es una descripción apoyada por la literatura citada. Sin tratar de resumir todo lo que se conoce del tema, ni de demostrar que se conoce toda la literatura. Hay que limitarse al tema específico del trabajo de investigación y a las contribuciones que se consideren más relevantes.

\section{Justificaci\'on (opcional)}\label{secc:jus}

En la justificación se debe expresar el por qué del estudio y la razón de su realización. Convencer al lector de que se hizo una investigación significativa: la importancia, la pertinencia del tema, el objeto de estudio y la utilidad de los resultados obtenidos.

\section{Hip\'otesis (opcional)}\label{secc:hipot}
La hipótesis del trabajo de tesis es la proposición que se pretende confirmar o refutar. No todas las investigaciones tienen hipótesis, sólo la necesitan aquellas que han rebasado la fase exploratoria. La hipótesis es la explicación que se le da a un hecho o fenómeno observado. Puede haber varias hipótesis para una misma pregunta de investigación y éstas no han de ser tomadas como verdaderas, sino que serán sometidas a pruebas para confirmar su veracidad.

\section{Objetivos}\label{secc:obj}
Los objetivos son las metas del conocimiento que se pretenden alcanzar, a qué resultados se quiere llegar. Es decir, son el destino de la tesis; el marco teórico, el terreno y la metodología, el camino a seguir. Los objetivos deben expresarse en forma concisa, clara e inequívoca. Se expresan comenzando con un verbo en infinitivo, por ejemplo: analizar, comparar, definir, clasificar, por mencionar algunos.
Los objetivos pueden dividirse en generales y específicos. El objetivo general es la descripción de la finalidad principal del estudio. Los objetivos específicos, si los hay, son considerados como secundarios. Son enunciados que facilitan la comprensión de las metas.
Los errores más comunes en la definición de los objetivos son:

\begin{itemize}
    \item Ser demasiado amplios y generalizados.
    \item Objetivos específicos no contenidos en los generales.
    \item Planteo de pasos como si fueran objetivos (confundirlos con métodos o metas).
		\item Confusión entre objetivos y políticas o planes para llegar a lo que es la finalidad práctica.
		\item Falta de relación entre los objetivos, el marco teórico y la metodología.
\end{itemize}

\subsection{Objetivo general}\label{sssec: objg}
\subsection{Objetivos especificos}\label{sssec: obje}

\newpage
%%=====================================================

\newpage }

{\normalsize
\chapter{Metodolog\'ia}\label{capit:cap2}
\vspace{-2.0325ex}%
\noindent
\rule{\textwidth}{0.5pt}
\vspace{-5.5ex}% 
\newcommand{\pushline}{\Indp}% Indent puede ir o no :p


En esta secc\'on se describen de forma detallada los procedimientos utilizados para la realizaci\'on de la investigaci\'on, con el propósito de que se pueda reproducir. Incluye una descripci\'on de los insumos utilizados por ejemplo: muestras colectadas, mediciones variables en el \'area de estudio o datos disponibles en alguna base de datos de acceso p\'ublico o privado. Esta secci\'on es apropiada para describir diseños experimentales, protocolos de adquisici\'on e instrumentaci\'on empleados. Se escribe en tiempo pasado y no debe ser una lista de materiales ni de pasos a seguir y es conveniente evitar el uso de t\'erminos ambiguos tales como: frecuentemente, regularmente, aproximadamente.
\\

\section{Ecuaci\'on (ejemplo)}\label{secc:ejemploec}

\begin{equation}
\theta(t) = \theta(0) + Zt0 \theta(\tau)dt
\label{eq:ejem}
\end{equation}

\section{Figura (ejemplo)}\label{secc:ejemplofig}

El Centro de Investigación Científica y de Educación Superior de Ensenada (CICESE) fue la segunda institución creada por el Consejo Nacional de Ciencia y Tecnología (CONACYT) para descentralizar las actividades científicas y tecnológicas en México.” (CICESE, 2019). En la Figura \ref{fig:ejemplo1} se puede ver el logo de CICESE. 


\begin{figure}[h]
        \centering
        \includegraphics[width=100mm]{./figuras/logoCicese2009.pdf}
        \caption{Logo de CICESE esta es una prueba que debe de permitir que se tengas mas de dos lienes en al tabla de las figuras} 
				\label{fig:ejemplo1}
\end{figure}

\section{Tablas (ejemplo)}\label{secc:ejemplotab}

		

\newpage
%%=====================================================

\newpage }

{\normalsize
\chapter{Resultados}\label{capit:cap3}
\vspace{-2.0325ex}%
\noindent
\rule{\textwidth}{0.5pt}
\vspace{-5.5ex}% 
\newcommand{\pushline}{\Indp}% Indent puede ir o no :p

Esta es la sección más importante de la tesis. Se sugiere presentar los resultados en el mismo orden que la metodología. Se describen con la ayuda de tablas, gráficas o figuras, de manera que faciliten su comprensión. Las tablas se utilizan generalmente para comunicar valores concretos de los datos, mientras que las figuras son mejores para mostrar tendencias o relaciones entre variables. Se recomienda utilizar subtítulos para agrupar resultados similares y para separar resultados de diferentes parámetros. Una regla básica es no duplicar la información de una tabla en gráficas. Si los mismos resultados de una tabla o figura dentro del contenido de la tesis proporcionan información relevante, se recomienda ponerla como anexo. 

\begin{table}[ht]
\centering
\resizebox{\textwidth}{!}{
\begin{tabular}{||c | c | c ||}
\hline
\hline
Categor\'ia  & Comportamiento  & Definici\'on  \\
\hline
\multirow{5}{4cm}{Actividad fisca. \\ Los movimientos que el niño realiza durante el juego} & Estacionario  & Mantenerse inmóvil durante 3 segundos o
más. Movimientos de dedos o pies\\
\cline{2-3}
& Movimiento extremidades & Movimientos del tronco, brazos y piernas sin mover todo el cuerpo de un lugar a otro\\
\cline{2-3}
& Translocacion lento & Mover el cuerpo de un lugar a otro con una velocidad lenta.\\
\cline{2-3}
& Translocacion medio & Mover el cuerpo de un lugar a otro con una velocidad moderada.\\
\cline{2-3}
& Translocacion ra\'pido & Mover el cuerpo de un lugar a otro con una velocidad rápida\\
\hline
\end{tabular}
}
\caption{Esquema de codificación.}
\label{codschema}
\end{table}

	
\newpage
%%=====================================================


\newpage }

{\normalsize

\chapter{Discusi\'on}\label{capit:cap4}
\vspace{-2.0325ex}%
\noindent
\rule{\textwidth}{0.5pt}
\vspace{-5.5ex}% 
\newcommand{\pushline}{\Indp}% Indent puede ir o no :p


En esta sección se deben interpretar y contrastar los resultados obtenidos. Se recomienda evitar demasiadas citas, ya que dificulta la lectura; hasta tres citas de fuentes formales son suficientes para respaldar cualquier afirmación. Es recomendable discutir las limitaciones del trabajo y de los métodos utilizados para minimizar o compensar esas limitaciones. Es importante enfatizar las implicaciones de los resultados y trazar futuras líneas de investigación.





\newpage
%%=====================================================

\newpage }

{\normalsize

\chapter{Conclusiones}\label{capit:cap5}
\vspace{-2.0325ex}%
\noindent
\rule{\textwidth}{0.5pt}
\vspace{-5.5ex}% 
\newcommand{\pushline}{\Indp}% Indent puede ir o no :p


Esta sección refuerza lo expuesto en la introducción, se exponen de manera concisa los hallazgos más importantes de la investigación, contrastándolos con las hipótesis y objetivos planteados al inicio del trabajo. Se recomienda agregar las aportaciones, las limitaciones que se tuvieron y recomendaciones para trabajo futuro de la investigación. No se deben mencionar elementos no estudiados en la investigación ni repetir el contenido de la tesis. 




\newpage
%%=====================================================

\newpage }


\linespread{1.0}
\addcontentsline{toc}{chapter}{\normalsize\expandafter{Literatura citada} \protect\texorpdfstring{\normalfont\protect\dotfill}{}}
{\normalsize
\bibliographystyle{cicese} 
\nocite{*}
\bibliography{secciones/literaturaTesis}
}

\linespread{1.5}
{\normalsize
\anexo{}


\chapter{Anexo} \label{chap:anexA}
Los anexos son secciones relativamente independientes que permiten conocer más a fondo aspectos específicos, que por su longitud o naturaleza no conviene incluir dentro del documento principal. Son elementos para dar una información más completa y que es útil para investigaciones futuras. Los anexos constituyen una sección adicional a la organización del trabajo y en ellos debe incluirse material complementario como: estadísticas, gráficas, fotografías, mapas, tablas, programas de cómputo, etcétera. Si no se menciona en el texto principal no deben incluirse. La información de los anexos debe ser completa, de manera que pueda utilizarse de forma independiente.

De acuerdo con las características de la información, el formato es libre. Se recomienda colocarlos en el orden en que están citados en el texto y, de preferencia, usando letras mayúsculas (Anexo A, Anexo B, Anexo C, etc.) 

\newpage }

\end{document}
