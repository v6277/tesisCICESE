\chapter{Introducci\'on esta es una prueba para que el título se pueda hacer en dos renglones}\label{capit:cap1}
\vspace{-2.0325ex}%
\noindent
\rule{\textwidth}{0.5pt}
\vspace{-5.5ex}% 
\newcommand{\pushline}{\Indp}% Indent puede ir o no :p

La introducción es la presentación del trabajo, informa sobre tres elementos muy importantes de la investigación: el propósito, la importancia del trabajo realizado y el conocimiento actual del tema. El texto debe comenzar con consideraciones generales y se recomienda terminar con el propósito del trabajo. Es conveniente que sea breve (no mayor de cinco páginas) y debe existir coherencia entre las distintas secciones que se presentan. La redacción debe ser clara, directa y sencilla, de tal forma que un lector no familiarizado con el tema pueda comprender el alcance del trabajo y motive su lectura.

\section{Antecedentes}\label{secc:antece}
En esta sección se profundiza en el conocimiento acerca del tema y la relación con el trabajo de investigación. Es una descripción apoyada por la literatura citada. Sin tratar de resumir todo lo que se conoce del tema, ni de demostrar que se conoce toda la literatura. Hay que limitarse al tema específico del trabajo de investigación y a las contribuciones que se consideren más relevantes.

\section{Justificaci\'on (opcional)}\label{secc:jus}

En la justificación se debe expresar el por qué del estudio y la razón de su realización. Convencer al lector de que se hizo una investigación significativa: la importancia, la pertinencia del tema, el objeto de estudio y la utilidad de los resultados obtenidos.

\section{Hip\'otesis (opcional)}\label{secc:hipot}
La hipótesis del trabajo de tesis es la proposición que se pretende confirmar o refutar. No todas las investigaciones tienen hipótesis, sólo la necesitan aquellas que han rebasado la fase exploratoria. La hipótesis es la explicación que se le da a un hecho o fenómeno observado. Puede haber varias hipótesis para una misma pregunta de investigación y éstas no han de ser tomadas como verdaderas, sino que serán sometidas a pruebas para confirmar su veracidad.

\section{Objetivos}\label{secc:obj}
Los objetivos son las metas del conocimiento que se pretenden alcanzar, a qué resultados se quiere llegar. Es decir, son el destino de la tesis; el marco teórico, el terreno y la metodología, el camino a seguir. Los objetivos deben expresarse en forma concisa, clara e inequívoca. Se expresan comenzando con un verbo en infinitivo, por ejemplo: analizar, comparar, definir, clasificar, por mencionar algunos.
Los objetivos pueden dividirse en generales y específicos. El objetivo general es la descripción de la finalidad principal del estudio. Los objetivos específicos, si los hay, son considerados como secundarios. Son enunciados que facilitan la comprensión de las metas.
Los errores más comunes en la definición de los objetivos son:

\begin{itemize}
    \item Ser demasiado amplios y generalizados.
    \item Objetivos específicos no contenidos en los generales.
    \item Planteo de pasos como si fueran objetivos (confundirlos con métodos o metas).
		\item Confusión entre objetivos y políticas o planes para llegar a lo que es la finalidad práctica.
		\item Falta de relación entre los objetivos, el marco teórico y la metodología.
\end{itemize}

\subsection{Objetivo general}\label{sssec: objg}
\subsection{Objetivos especificos}\label{sssec: obje}

\newpage
%%=====================================================
